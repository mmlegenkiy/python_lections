\section{Лекція 5: Умовні оператори}
 
 \subsection{Що таке цикл?} 
\begin{frame}
% \frametitle{Логічні висновки}
Цикли — оператори, які дозволяють повторювати код певну кількість разів.

\huge{цикл (умова):

~~~~операції
}

\normalsize Позначення:

\large{заголовок циклу:

~~~~тіло циклу
}


\end{frame}

\begin{frame}
\frametitle{Цикл while}
Визначимо суму чисел від 1 до N.

s = 0

i = 1 

N = 1000

while i <= N:

~~~~s += i

~~~~i += 1

Одноразове виконання тіла цикла називається ітерацією цикла.

\end{frame}

 \subsection{Оператори break та continue} 
\begin{frame}
% \frametitle{Цикл while}
\begin{itemize}
  \item break - дострокове завершення циклу.
  \item continue - пропуск однієї ітерації циклу.
  \item В блоці else йдуть команди, що виконуються після штатного завершення циклу.
\end{itemize}
\end{frame}
