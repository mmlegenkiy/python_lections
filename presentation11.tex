\section{Лекція 11: Робота з файлами}
 
 \subsection{Функція open} 
\begin{frame}
% \frametitle{Логічні висновки}
Для того, щоб отримати доступ до файлу використовується функція \texttt{open}:

\texttt{f = open(file[, mode='r', encoding=None, ...])}

\begin{itemize}
  \item \texttt{file} - шлях до файлу;
  \item \texttt{mode} - режим доступу до файлу (читання/запис);
  \item \texttt{encoding} - кодування файлу.
\end{itemize}

\texttt{f.read()} - читає весь вміст файлу \texttt{f}. 
\end{frame}

\begin{frame}
% % \frametitle{Логічні висновки}
\texttt{f.read(4)} - читає перші 4 символи файлу \texttt{f} (читання розпочинається з місця, на яке вказує файлова позиція). Файлова позиція вказується в байтах.

Для встановлення файлової позиції в положення \texttt{offset} використовується функція \texttt{f.seek(offset[, from\_what])}.

Метод \texttt{f.tell()} повертає поточну файлову позицію.

Метод \texttt{f.readline()} читає рядок із файлу.
%
\texttt{f} сприймається як ітерований об'єкт і при ітерації повертає новий прочитаний рядок із файлу.

Метод \texttt{f.readlines()} читає вміст файлу та повертає список із рядків файлу.

\texttt{f.close()} закриває файл.
\end{frame}

\begin{frame}
% % \frametitle{Логічні висновки}
Якщо файл не знайдено виникає помилка \texttt{FileNotFoundError}. Для обробки таких та подібних виключень використовується конструкція:

\texttt{try:}

\texttt{~~~~operators\_1}

\texttt{except [виключення]:}

\texttt{~~~~operators\_2}

\texttt{finally:}

\texttt{~~~~operators\_3}

Блок \texttt{finally} виконується в будь-якому випадку якщо трапилось виключення, або якщо не трапилось.

\end{frame}

\begin{frame}
% % \frametitle{Логічні висновки}
Уникнути зупинки програми у випадку помилки при роботі з файлом можна за допомогою менеджера контексту:

\texttt{with open(file\_path) as file:}

\texttt{~~~~operators}

Виконання блоку \texttt{with} завжди завершується закриттям файлу.

\end{frame}

\begin{frame}
% % \frametitle{Логічні висновки}
\begin{itemize}
  \item \texttt{open(file\_path)} або \texttt{open(file\_path,"r")} - відкриття файлу для читання.
  \item \texttt{open(file\_path,"w")} - відкриття файлу для запису.
  \item \texttt{open(file\_path,"a")} - відкриття файлу для дозапису.
  \item \texttt{open(file\_path,"a+")} - відкриття файлу для дозапису та читання.
\end{itemize}
Для запису рядка \texttt{string} до файлу використовується метод \texttt{file.write(string)}. Метод \texttt{file.writelines(lst)} записує відразу декілька рядків до файлу, кожен рядок відповідає елементу списка \texttt{lst}.
\end{frame}

\begin{frame}
% % \frametitle{Логічні висновки}
Можна працювати з бінарними файлами.
\begin{itemize}
  \item \texttt{open(file\_path,"rb")} - відкриття бінарного файлу для читання.
  \item \texttt{open(file\_path,"wb")} - відкриття бінарного файлу для запису.
\end{itemize}
Для роботи з бінарними файлами використовується модуль \texttt{pickle}: \texttt{import pickle}.
Для запису до файлу \texttt{file} даних \texttt{data} у бінарному вигляді використовується метод \texttt{pickle.dump(data, file)}. Для зчитування даних із бінарного файлу \texttt{file} використовується метод \texttt{pickle.load(file)}.
\end{frame}
