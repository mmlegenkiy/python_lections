\section*{Лекція 11: Робота з файлами}
 
 \subsection{Читання та запис файлу} 
\begin{frame}
\frametitle{Функція open}
Для того, щоб отримати доступ до файлу використовується функція \texttt{open}:

\texttt{f = open(file[, mode='r', encoding=None, ...])}

\begin{itemize}
  \item \texttt{file} - шлях до файлу;
  \item \texttt{mode} - режим доступу до файлу (читання/запис);
  \item \texttt{encoding} - кодування файлу.
\end{itemize}

Після закінчення роботи з файлом його обов'язково слід закрити, застосувавши метод \texttt{f.close()}.

\end{frame}

\begin{frame}
\frametitle{Методи роботи з файлом f}

\texttt{f} сприймається як ітерований об'єкт і при ітерації повертає новий прочитаний рядок із файлу.

\begin{itemize}
  \item<1->  \texttt{f.read()} - читає весь вміст файлу \texttt{f}. 
  \item<1-> \texttt{f.readlines()} - читає вміст файлу та повертає список із рядків файлу. 
  \item<2-> \texttt{f.readline()} - читає рядок із файлу.
  \item<2-> \texttt{f.read(n)} - читає перші \texttt{n} символів файлу \texttt{f} (читання розпочинається з місця, на яке вказує файлова позиція). Файлова позиція вказується в байтах.
  
\end{itemize}
\end{frame}

\begin{frame}
\frametitle{Методи роботи з файлом f}

\texttt{f} сприймається як ітерований об'єкт і при ітерації повертає новий прочитаний рядок із файлу.

\begin{itemize}
  \item<1->  \texttt{f.seek(offset[, from\_what])} - встановлення файлової позиції в положення \texttt{offset}.
  \item<1-> \texttt{f.tell()} - повертає поточну файлову позицію.
  \item<2-> \texttt{f.write(string)} - запис рядка \texttt{string} до файлу. Метод      
  \item<2-> \texttt{f.writelines(lst)} - запис декількох рядків до файлу, кожен рядок відповідає елементу списка \texttt{lst}.
\end{itemize}

\end{frame}

\begin{frame}
\frametitle{Способи відкриття файлу}
\begin{itemize}
  \item<1-> \texttt{open(file\_path)} або \texttt{open(file\_path,"r")} - відкриття файлу для читання.
  \item<2-> \texttt{open(file\_path,"w")} - відкриття файлу для запису.
  \item<3-> \texttt{open(file\_path,"a")} - відкриття файлу для дозапису.
  \item<4-> \texttt{open(file\_path,"a+")} - відкриття файлу для дозапису та читання.
  \item<5-> \texttt{open(file\_path,"rb")} - відкриття бінарного файлу для читання.
  \item<5-> \texttt{open(file\_path,"wb")} - відкриття бінарного файлу для запису.
\end{itemize}

\end{frame}

\subsection{Обробка помилок}

\begin{frame}
\frametitle{Конструкція try-except}
Якщо файл не знайдено виникає помилка \texttt{FileNotFoundError}. Для обробки таких та подібних виключень використовується конструкція:

\texttt{try:}

\texttt{~~~~operators\_1}

\texttt{except [виключення]:}

\texttt{~~~~operators\_2}

\texttt{finally:}

\texttt{~~~~operators\_3}

Блок \texttt{finally} виконується в будь-якому випадку.

\end{frame}

\begin{frame}
\frametitle{Менеджер контексту}
Уникнути зупинки програми у випадку помилки при роботі з файлом можна за допомогою менеджера контексту:

\vspace{0.5cm}

\Large
\texttt{with open(file\_path) as file:}

\texttt{~~~~operators}

\vspace{0.5cm}
\normalsize
Виконання блоку \texttt{with} завжди завершується закриттям файлу.

\end{frame}

\subsection{Робота з бінарними файлами}
\begin{frame}
\frametitle{Модуль pickle}
Для роботи з бінарними файлами використовується модуль \texttt{pickle}: \texttt{import pickle}.

Для роботи із бінарними файлами їх треба відкрити для читання або запису в бінарному вигляді:
\texttt{open(file\_path,"rb")} або  \texttt{open(file\_path,"wb")}.

\begin{itemize}
  \item \texttt{pickle.dump(data, file)} - запис до файлу \texttt{file} даних \texttt{data} у бінарному вигляді.
  \item  \texttt{pickle.load(file)} - зчитування даних із бінарного файлу \texttt{file}.
\end{itemize}

\end{frame}
