% \documentclass[t]{beamer}  % [t], [c], или [b] --- вертикальное выравнивание на слайдах (верх, центр, низ)
%\documentclass[handout]{beamer} % Раздаточный материал (на слайдах всё сразу)
%\documentclass[aspectratio=169]{beamer} % Соотношение сторон

\usetheme{Berlin} % Тема оформления
%\usetheme{Madrid}
%\usetheme{Frankfurt}
%\usetheme{CambridgeUS}

\usecolortheme{albatross} % Цветовая схема
%\usecolortheme{monarca} % Цветовая схема
%\usecolortheme{fly} % Цветовая схема
%\useinnertheme{circles}
%\useinnertheme{rectangles}

%\usetheme{HSE}

%%% Работа с русским языком
\usepackage{cmap}					% поиск в PDF
\usepackage{mathtext} 				% русские буквы в формулах
\usepackage[T2A]{fontenc}			% кодировка
% \usepackage[utf8]{inputenc}			% кодировка исходного текста
\usepackage[english,ukrainian]{babel}	% локализация и переносы
\usepackage{amsmath}
\usepackage{amsfonts}
\usepackage{amssymb}

\usepackage{hyperref}

%%% Работа с картинками
\usepackage{graphicx}  % Для вставки рисунков
\graphicspath{{images/}{images2/}}  % папки с картинками
\setlength\fboxsep{3pt} % Отступ рамки \fbox{} от рисунка
\setlength\fboxrule{1pt} % Толщина линий рамки \fbox{}
\usepackage{wrapfig} % Обтекание рисунков текстом

%%% Работа с таблицами
\usepackage{array,tabularx,tabulary,booktabs} % Дополнительная работа с таблицами
\usepackage{longtable}  % Длинные таблицы
\usepackage{multirow} % Слияние строк в таблице

%%% Программирование
\usepackage{etoolbox} % логические операторы

%%% Другие пакеты
\usepackage{lastpage} % Узнать, сколько всего страниц в документе.
\usepackage{soul} % Модификаторы начертания
\usepackage{csquotes} % Еще инструменты для ссылок
%\usepackage[style=authoryear,maxcitenames=2,backend=biber,sorting=nty]{biblatex}
\usepackage{multicol} % Несколько колонок

%%% Картинки
\usepackage{tikz} % Работа с графикой
\usepackage{pgfplots}
\usepackage{pgfplotstable}

\title[Python]{Мова програмування Python для початківців}
%\subtitle{За матеріалами "Системне адміністрування Linux" Сергія Клочкова}
\author{Легенький М.М.}
\date{\today}
\institute[Факультет радіофізики, біомедичної електроніки та комп'ютерних систем]{Факультет радіофізики, біомедичної електроніки та комп'ютерних систем}

\setbeamertemplate{caption}[numbered]

\titlegraphic { 
\begin{tikzpicture}[overlay,remember picture]
\node[left=0.2cm] at (current page.32){
    \includegraphics[width=3cm]{rbecs_logo}
};
\node[right=0cm] at (current page.148){
    \includegraphics[width=1.5cm]{python_logo}
};
\end{tikzpicture}
}

\logo{
\begin{tikzpicture}[overlay,remember picture]
\node[left=0cm] at (current page.-26){
    \includegraphics[width=3cm]{rbecs_logo}
};
\node[right=0cm] at (current page.-153){
    \includegraphics[width=1.5cm]{python_logo}
};
\end{tikzpicture}}
