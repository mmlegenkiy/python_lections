\section*{Лекція 16: Застосування Python. Створення Telegram бота}

\subsection{Знайомство з pyTelegramBotApi} 

\begin{frame}
\frametitle{Створення бота через BotFather}

Спочатку треба створити бота за допомогою \href{https://t.me/BotFather}{BotFather}. Після команди \texttt{/newbot} треба встановити \texttt{name} та \texttt{username} для боту і отримати \texttt{API Token}.
 
\begin{figure}
  \begin{center}
    \includegraphics[width=0.25\textwidth,height=0.4\textheight]{pictures/botFather.png}
  \caption{\href{https://t.me/BotFather}{BotFather}}
\label{function}
  \end{center}
\end{figure}
\end{frame}

\begin{frame}
\frametitle{Луна-бот}

 Для взаємодії з ботом через HTTP-запити використовується бібліотека pyTelegramBotAPI:
\texttt{pip install pyTelegramBotAPI}.

Найпростіший бот:

\texttt{import telebot}

\texttt{bot = telebot.TeleBot(API\_TOKEN)}

\texttt{@bot.message\_handler(commands=['start'])}

\texttt{def start(message):}

\texttt{~~~~bot.send\_message(message.chat.id, message.text)}

\texttt{bot.polling()}

\end{frame}

\subsection{Створення простого бота}
\begin{frame}
\frametitle{Робота з ботом в Python}

https://mastergroosha.github.io/telegram-tutorial/

\end{frame}

\frametitle{Робота з ботом в Python}

https://mastergroosha.github.io/telegram-tutorial/

\end{frame}
