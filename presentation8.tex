\section{Лекція 8: Генератори}
 
 \subsection{Що таке генератор?} 
\begin{frame}
% \frametitle{Логічні висновки}
У Python є спеціальна синтаксична конструкція, яка дозволяє за певними правилами створювати заповнені списки. Такі конструкції називають генераторами списків (list comprehension). Їх зручність полягає у більш короткому запису програмного коду, ніж якби створювався список звичайним способом.

\LARGE{[правило for змінна in ітератор]}

\normalsize або

\LARGE{[правило for змінна in ітератор if умова]}

\normalsize Наприклад:

[i**2 for i in range(N)]

\end{frame}

\begin{frame}
% \frametitle{Логічні висновки}
Генератори списків можуть бути вкладені.

\LARGE{[правило

 for змінна\_1 in ітератор\_1 if умова\_1
 
 ...
 
 for змінна\_N in ітератор\_N if умова\_N]}

\normalsize Така конструкція працює так само як низка вкладених циклів \texttt{for}.

\end{frame}

\begin{frame}
% \frametitle{Логічні висновки}
В якості \textit{правила} в генераторі списку можна використовувати ще один генератор списку.

\Large{[[ правило for змінна\_1 in ітератор\_1 if умова\_1]

 for змінна\_2 in ітератор\_2 if умова\_2]}

\end{frame}

\begin{frame}
% \frametitle{Логічні висновки}
Вкладений генератор списку можна використовувати для транспонування прямокутної матриці A:

\vspace{1cm}

\large{[[ row[i] for row in A] for i in range(len(A[0]))]}

\end{frame}

\begin{frame}
% \frametitle{Логічні висновки}
Генератор списку може використовуватися як ітератор для генератору списку.

\vspace{1cm}

\LARGE{[правило\_2 for змінна\_2 in 

[правило\_1 for змінна\_1 in ітератор]]}

\vspace{1cm}

\normalsize Такий генератор працює як вкладена фукнція.

\end{frame}

 \subsection{Генератор множин} 
\begin{frame}
% \frametitle{Логічні висновки}
Аналогічно до генераторів списків можна побудувати генератор множин.

\LARGE{\{правило for змінна in ітератор\}}

\normalsize або

\LARGE{\{правило for змінна in ітератор if умова\}}

\end{frame}

 \subsection{Генератор словника} 
\begin{frame}
% \frametitle{Логічні висновки}
Словник можна отримати за допомогою генератора, аналогічного до генератору списку або множини.

\Large{\{ключ: значення for змінна in ітератор\}}

\normalsize або

\Large{\{ключ: значення for змінна in ітератор if умова\}}

\end{frame}
