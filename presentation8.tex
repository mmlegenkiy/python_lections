\section*{Лекція 8: Генератори}
 
 \subsection{Генератор списку} 
\begin{frame}
\frametitle{Що таке генератор?}
Генератор списку (list comprehension) - спеціальна синтаксична конструкція, яка дозволяє за певними правилами створювати заповнені списки. Зручність: короткий запис програмного коду.

\LARGE{[правило for змінна in ітератор]}

\normalsize або

\LARGE{[правило for змінна in ітератор if умова]}

\normalsize Наприклад:

[i**2 for i in range(N)]

\end{frame}

\begin{frame}
\frametitle{Вкладені генератори списків}
Генератори списків можуть бути вкладені.

\LARGE{[правило

 for змінна\_1 in ітератор\_1 if умова\_1
 
 ...
 
 for змінна\_N in ітератор\_N if умова\_N]}

\normalsize Така конструкція працює так само як низка вкладених циклів \texttt{for}.

\end{frame}

\begin{frame}
\frametitle{Вкладені генератори списків}
В якості \textit{правила} в генераторі списку можна використовувати ще один генератор списку.

\Large{[[ правило for змінна\_1 in ітератор\_1 if умова\_1]

 for змінна\_2 in ітератор\_2 if умова\_2]}

\end{frame}

\begin{frame}
\frametitle{Використання генератора списку}
Вкладений генератор списку можна використовувати для транспонування прямокутної матриці A:

\vspace{1cm}

\large{[[ row[i] for row in A] for i in range(len(A[0]))]}

\end{frame}


\begin{frame}
\frametitle{Використання генератора списку}
Генератор списку може використовуватися як ітератор для генератору списку.

\vspace{1cm}

\LARGE{[правило\_2 for змінна\_2 in 

[правило\_1 for змінна\_1 in ітератор]]}

\vspace{1cm}

\normalsize Такий генератор працює як вкладена фукнція.

\end{frame}

\subsection{Генератор множин та словників}
\begin{frame}
\frametitle{Що таке генератор множин?}
Аналогічно до генераторів списків можна побудувати генератор множин.

\LARGE{\{правило for змінна in ітератор\}}

\normalsize або

\LARGE{\{правило for змінна in ітератор if умова\}}

\end{frame}

 \begin{frame}
\frametitle{Що таке генератор словників?}
Словник можна отримати за допомогою генератора, аналогічного до генератору списку або множини.

\Large{\{ключ: значення for змінна in ітератор\}}

\normalsize або

\Large{\{ключ: значення for змінна in ітератор if умова\}}

\end{frame}

\begin{frame}
\frametitle{Внутрішній генератор}
Для генераторів списків, множин та словників працює внутрішній генератор, що створює елементи цих об'єктів:
\Large{(правило for змінна in ітератор)}

\normalsize або

\Large{(правило for змінна in ітератор if умова)}

\normalsize Генератор перебирають тільки один раз. Функції list, set, tuple, sum, max, min приймають як аргумент генератор. Генератори не зберігають значення, а створюють їх за потреби.


\end{frame}

\subsection{Функція-генератор} 
\begin{frame}
\frametitle{Оператор yield}
Щоб перетворити функцію в функцію-генератор слід замість оператору \texttt{return} використовувати оператор \texttt{yield}.

\texttt{def numbers\_range(n):}

~~~~\texttt{for i in range(n):}

~~~~~~~~\texttt{yield i}   
        
\texttt{a = numbers\_range(4)}

\texttt{print(type(a))}

\texttt{for b in a:}

~~~~\texttt{print(b)}
\end{frame}

\begin{frame}
\frametitle{Фукнція map}
Фукнція \texttt{map} застосовує функцію \texttt{func} до кожного елементу ітератору \texttt{it}, як результат отримується генератор \texttt{gen}: 

\texttt{gen = map(func, it)}. 

Функція map еквівалентна генератору:

\texttt{(func(x) for x in it)}

Всередині фукнції \texttt{map} можна використовувати \texttt{lambda}-функцію. 
\end{frame}

\begin{frame}
\frametitle{Фукнція filter}
Фукнція \texttt{filter} слугує для фільтрації елементів ітерованого об'єкту.

\texttt{filter(func, it)}

Якщо функція \texttt{func} повертає значення \texttt{True}(\texttt{False}) для поточного значення ітерованого об'єкту, то це значення повертається (не повертається).

Всередині фукнції \texttt{filter} можна використовувати \texttt{lambda}-функцію. Як ітерований об'єкт \texttt{it} всередині функції \texttt{filter}  можна використовувати результат роботи іншої функції \texttt{filter}.
\end{frame}

\begin{frame}
\frametitle{Фукнція zip}
Фукнція \texttt{zip(it\_1[, it\_2 , it\_3])} для вказаних ітерованих об'єктів здійснює відповідних перебір і продовжує роботу доки не дійде до кінця самої короткої колекції. В результаті отримуємо ітератор кортежів.

Приклад:

\texttt{s = 'abc'}

\texttt{t = (10, 20, 30)}

\texttt{list(zip(s,t))}

[('a', 10), ('b', 20), ('c', 30)]
\end{frame}
