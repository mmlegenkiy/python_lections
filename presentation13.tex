\section{Лекція 13: Модулі та імпорт}
 
 \subsection{Функція import} 
\begin{frame}
% \frametitle{Логічні висновки}
Команда \texttt{import module} додає до глобального простору простор імен, що містить визнчені в модулі змінні, функції, класи. 

Python має велику кількість стандартних бібліотек: \href{https://docs.python.org/3/library/}{https://docs.python.org/3/library/}.

Щоб вказати псевдонім \texttt{mod} модуля \texttt{module} використовується команда \texttt{import module as mod}.

Щоб виконати вибірковий імпорт функції \texttt{function} (з псевдонімом \texttt{fun}) із модуля \texttt{module} використовується команда \texttt{from module import function as fun}.

\texttt{from module import *} - імпортувати все.

\end{frame}

\begin{frame}
% \frametitle{Логічні висновки}
В списку \texttt{sys.path} містеться перелік шляхів, за якими шукає Python модулі. Якщо модуль міститься в підкаталозі по відношенню до поточного (робочого), то його назву треба вказувати через крапку після назви каталогу. 

В кожному модулі є змінна \texttt{\_\_name\_\_}. Якщо даний модуль запускається на виконання, то ця змінна дорівнюватиме \texttt{\_\_main\_\_}, інакше (якщо модуль імпортується) вона дорівнюватиме назві модуля.

Щоб подивитись список встановлених пакетів виконайте команду \texttt{pip list} в терміналі. Ознайомитись зі списком існуючих пактів можна за посиланням \href{https://pypi.org/}{https://pypi.org/}.
\end{frame}

\subsection{Зовнішні пакети Python} 
\begin{frame}
% \frametitle{Логічні висновки}
\begin{itemize}
  \item NumPy - робота з багатовимірними масивами;
  \item Matplotlib - відображення графіків;
  \item Pygame - реалізація ппростої 2D-графіки;
  \item Flask - простий фреймворк (часто для інформаційних сайтів);
  \item Django - просунутий фреймворк для складних сайтів;
\end{itemize}

Для встановлення пакету \texttt{pygame} виконується команда \texttt{pip install pygame}. Створення списку потрібних пакетів \texttt{pip freeze > requirements.txt}. Пакетне встановлення зовнішніх модулів \texttt{pip install -r requirements.txt}. 
\end{frame}

\begin{frame}
% \frametitle{Логічні висновки}
Пакет (package) - спеціальним чином організований підкаталог з низкою модулів, що зазвичай розв'язують схожі задачі. Пакет - це директорія з файлом \_\_init.py\_\_.

Пакети імпортуються так само як і модулі \texttt{import package}. Все, що треба імпортувати при імпорті пакету вказано в файлі \_\_init.py\_\_. В цьому файлі щоб зробити доступною функцію \texttt{function} з модулю \texttt{module}  пишеться команда \texttt{from .module import function}.

Змінна \_\_all\_\_ містить список функції, що імпортуються при імпорті модуля.
\end{frame}
