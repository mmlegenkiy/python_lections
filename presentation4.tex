\section{Лекція 4: Списки}
\subsection{Створення списків в Python} 
\begin{frame}
% \frametitle{Створення рядків в Python}
Список - впорядкована колекція даних різних типів.

Список відноситься до змінних типів даних.

Пустий список - []. Для створення списку на основі об'єкту, який можна перебирати, використовується функція list.
\begin{figure}
\begin{center}
 \includegraphics[width=0.8\textwidth]{pictures/list.png}
\caption{Список}
\label{list} 
\end{center}
\end{figure}

\end{frame}

\subsection{Основні  функції для роботи зі списком s} 
\begin{frame}
    \begin{itemize}
        \item<1-> \texttt{len(s)} - визначення числа елементів у списку;
        \item<2-> \texttt{max(s)} - знаходження максимального значення;
        \item<2-> \texttt{min(s)} - знаходження мінімального значення;
        \item<3-> \texttt{sum(s)} - розрахунок суми;
        \item<4-> \texttt{sorted(s)} - сортування за зростанням;
        \item<4-> \texttt{sorted(s, reverse=True)} - сортування за спаданням.
    \end{itemize}
\end{frame}

\subsection{Оператори для роботи зі списками} 
\begin{frame}
    \begin{itemize}
        \item<1-> \texttt{+} - поєднання двох списків в один;
        \item<2-> \texttt{*} - повторення списку;
        \item<3-> \texttt{in} - перевірка входження елементу в список;
        \item<4-> \texttt{del} - видалення елементу списку.
    \end{itemize}
\end{frame}

% \subsection{Оператори для роботи зі списками} 
\begin{frame}
Зрізи дозволяють отримати деяку підмножину значень списку:

\begin{center}
\huge{lst[start:stop[:step]]}
\end{center}
\normalsize

Для створення копії списка \texttt{lst} використовується команда \texttt{lst[:]} або \texttt{list(lst)}.

Списки можна порівнювати між собою за допомогою операторів: >, <, == та !=.
\end{frame}

\subsection{Методи списку lst} 
\begin{frame}
\begin{center}
об'єкт.метод(аргументи)
\end{center}
\begin{itemize}
        \item<1-> \texttt{lst.append(el)} - додати елемент el в кінець списку lst;
        \item<2-> \texttt{lst.insert(pos, el)} - додати елемент el до списку lst на місце з індексом pos;
        \item<3-> \texttt{lst.remove(val)} - видаляє перший елемент val зі списку lst (якщо елементу немає в списку, отримуємо помилку);
        \item<4-> \texttt{lst.pop()} - видаляє та \textbf{повертає} останній елемент зі списку lst;
        \item<4-> \texttt{lst.pop(ind)} - видаляє та \textbf{повертає} елемент з індексом ind зі списку lst.
    \end{itemize}
\end{frame}

\subsection{Методи списку lst} 
\begin{frame}
\begin{center}
об'єкт.метод(аргументи)
\end{center}
\begin{itemize}
        \item<1-> \texttt{lst.clear()} - видаляє всі елементи зі списку lst;
        \item<2-> \texttt{lst.copy()} - повертає копію списку lst;
        \item<3-> \texttt{lst.count(val)} - знаходить число елементів зі значенням val в списку lst;
        \item<4-> \texttt{lst.index(val)} - знаходить індекс значення val в списку lst;
        \item<4-> \texttt{lst.index(val, start)} - знаходить індекс значення val в списку lst, починаючи з індексу start;
    \end{itemize}
\end{frame}

\subsection{Методи списку lst} 
\begin{frame}
\begin{center}
об'єкт.метод(аргументи)
\end{center}
\begin{itemize}
        \item<1-> \texttt{lst.reverse()} - змінює порядок елементів списку lst на зворотній;
        \item<2-> \texttt{lst.sort()} - сортує елементи списку lst за зростанням;
        \item<2-> \texttt{lst.sort(reverse=True)} - сортує елементи списку lst за спаданням;
    \end{itemize}
\end{frame}

\subsection{Вкладені списки} 
\begin{frame}
\begin{center}
Вкладений список = двовимірний список.

lst = [line[:], line[:], line[:]]

Щоб отримати один елемент використовуємо команду lst[i][j]. 
\end{center}

\end{frame}
