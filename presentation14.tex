\section{Лекція 14: Об’єкти та класи}
\subsection{Функція isinstance} 
\begin{frame}
Функція \texttt{isinstance(obj, type)} перевіряє чи належить об'єкт \texttt{obj} до типу даних \texttt{type} і повертає \texttt{True} або \texttt{False}. Замість \texttt{type} можна використовувати кортеж із декількох типів даних.

Тип \texttt{bool} успадковується від типу \texttt{int}, тому ця функція показує, що значення \texttt{True} або \texttt{False} належать до типу \texttt{int}.

Перевірку типу даних також можна робити за допомогою функції \texttt{type}:

\texttt{type(a) == int} або \texttt{type(a) in (int, float)}  


% \frametitle{Створення рядків в Python}
\end{frame}

\begin{frame}
ООП - об'єктно орієнтоване програмування. В програмах ми оперуємо об'єктами даних.

Класи та об'єкти класів.

Клас містить властивості та методи.

Дані та методи класу можна приховувати. Інкапсуляція.

Успадкування. Властивості та методи успадковуються класами від базових класів. Дочірні класи розширюють функціональність базових.

Поліморфізм - можливість через единий інтерфейс працювати з об'єктами різних класів.
\end{frame}

\begin{frame}
Визначення класу.
Name - назва класу, attr\_1 та attr\_2 - атрибути класу. 

\texttt{class Name:}

\texttt{~~~~attr\_1}

\texttt{~~~~attr\_2}

 Фактично клас створює простір імен з назвою Name. Атрибути класу можна знайти в словнику \texttt{Name.\_\_dict\_\_}. Створення екземпляра класу \texttt{a = Name()}.
\end{frame}
